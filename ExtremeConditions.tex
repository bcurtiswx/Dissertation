\section[Extreme Conditions]{Extreme conditions in the magnetosphere}

\subsection{Background and Motivation}
In order to improve predictability of the magnetopause location under extreme
events, \citet{Shue1998} took magnetopause crossing satellite measurements and
compared them to two models. The first model (\textit{Petrinic and Russell},
1996) was compared to the \textit{Shue et. al.} (1997) model. Both models
compared well with the magnetopause crossings at the day-side magnetopause,
while the \textit{Shue et. al.} (1997) model had a poorer fit for magnetopause
crossings at the flanks.
The explanation for the discrepancies was that they were ``due to the inappropriate linear extrapolation from the parameter range for average solar wind conditions to that
for extreme conditions''. Upon correction, the \textit{Shue et. al.}
(1997) model was able to better predict magnetopause flank crossings.

Companies and government agencies with space-based assests are interested in the
duration of extreme storms. \citet{Cid2013} studied the effectiveness of a
hyperbolic function for estimating the decay time after minimum $D_{st}$ values
for extreme storms.
A hyperbolic function was used because previously used linear functions did not
accurately predict the the decay time of extreme storms. The extremity of the
storm was determined by the $D_{st}$ index where data was available, and a
``Local Disturbance Index'' taken from the $H$ component of geomagnetic field
measured at each observatory where $D_{st}$ data was not available.

Extreme space weather events are an active area of research.  For
example, statistical analysis on the long range correlations was by
\citet{Sharma2011} used a database of over 5 million events. The basis
for the research was that dynamical and statistical features in
extreme events are complicated due to the turbulent nature of the
solar wind.  An auto-correlation function and a detrended fluctuation
analysis were performed to find the long-range correlations.  In this
work, the extreme events were compiled from a
database. Although the used data was not model--based and the approach involved examining statistical properties, it did not involve a comparison to other models.
By comparing multiple models given the same generic input, as done in this experiment, along
with the extremes of input variables, forecasters may gain a better
understanding of which model is best to use for a variety of extreme
space weather conditions.

%\subsection{Motivation}
%Two of the validation techniques described by \citet{Sargent2004},
%\textit{comparison to other models}, and \textit{parameter variability} were us%ed in this section to better understand model behavior under extreme conditions.
%\citet{Cid2013} compared both a hyperbolic and linear fit to recovery times of
%strong geomagnetic storms to determine which provided the best comparison. The
%study determined that the hyperbolic fit was the best match for strong storms,
%and as the storm intensity decreased a linear approach became the better choice%.
%This study uses a \textit{historical data} validation.

\subsection{Methodology}
The similarities of the methodologies in all three experiments are described in
the methodology section for the $B_z^{IMF}$ reversal experiment in section
\ref{SimilarMethodology} of this dissertation. 

\begin{table}
\begin{center}
  \caption{Input parameters for extreme conditions experiment}
  \begin{tabular}{| l | c | c | c | c | }
    \hline
    \textbf{Run Num.} & \textbf{$\rho$} & \textbf{$T$} & \textbf{$Vx$} &
    \textbf{$B_z$}
    \\
    \hline 
    3 & 11 & 101289 & -604  & -3.0 \\ \hline
    4 & 2 & 101289 & -320  & 3.1 \\ \hline
  \end{tabular}
  \label{table:runs34}
\end{center}
\end{table}
As shown in Table \ref{table:runs34}, in order to compress the magnetosphere,
input variables were chosen corresponding to a high solar wind velocity, a negative
solar wind magnetic field, and a high solar wind density. Under high
compression, magnetospheric features may be difficult to resolve due to limitations in resolution.  For the low compression run, input variables were chosen that lead to a small
compression of the magnetosphere: a low solar wind velocity, a positive $B_z$,
and a low density.

\subsection{Results}
\subsubsection{High Magnetospheric Compression}
% B%
As shown in Figure \ref{fig:BHighCompressionBeginning}, all three models have
similar $B_z$ contours. The OpenGGCM model magnetopause is closest to Earth.
In Figure \ref{fig:BDiffHighBeginning}, the BATS-R-US model magnetopause is
shown to be farther Sunward than that for the SWMF model. Figure
\ref{fig:BDiffHighBeginning} also shows higher $B_z$ values
to occur in the tail region for the BATS-R-US and SWMF models.

\begin{figure}
	\centering
	\subfigure{%
		\includegraphics[scale=0.36]{/mnt/Disk2/Brian_Curtis_042413_1/Results/images/Bz_File9.png}
		}
	\subfigure{%
		\includegraphics[scale=0.36]{/mnt/Disk2/Brian_Curtis_042413_2/Results/images/Bz_File9.png}
		}
	\subfigure{%
		\includegraphics[scale=0.36]{/mnt/Disk2/Brian_Curtis_042413_3/Results/images/Bz_File9.png}
		}
	\caption{$B_z$ for OpenGGCM (top), BATS-R-US (middle), and SWMF (bottom).}
	\figSpace
	\label{fig:BHighCompressionBeginning}
\end{figure}

\begin{figure}
	\centering
	\subfigure{
		\includegraphics[scale=0.36]{/mnt/Disk2/Results/8_9/images/Bz_Diff_File9.png}
	}
	\subfigure{
		\includegraphics[scale=0.36]{/mnt/Disk2/Results/8_10/images/Bz_Diff_File9.png}
    }
    \subfigure{
		\includegraphics[scale=0.36]{/mnt/Disk2/Results/9_10/images/Bz_Diff_File9.png}
	}
    \caption{$B_z$ percent differences between OpenGGCM and BATS-R-US (top),
    OpenGGCM and SWMF (middle), and BATS-R-US and SWMF (bottom).
    }
    \label{fig:BDiffHighBeginning}
	\figSpace
\end{figure}

The shape of the BATS-R-US and SWMF magnetosphere at the end of the model runs have
not significantly changed from the beginning, as shown in Figure
\ref{fig:BHighCompressionEnd}. The OpenGGCM model has higher $B_z$ values in the
near-Earth current sheet region. The values of $B_z$, $\rho$ and $U_x$ stabilize to a near constant value approximately one hour into the BATS-R-US run. The BATS-R-US and SWMF models, 
compared in Figure \ref{fig:BDiffHighEnd}, show large regions of higher $B_z$ from the BATS-R-US
model that are next to regions with higher $B_z$ from the SWMF model. The SWMF model,
viewed from $U_x$ plots, shows the largest oscillations.
\begin{figure}
	\centering
	\subfigure{%
		\includegraphics[scale=0.36]{/mnt/Disk2/Brian_Curtis_042413_1/Results/images/Bz_File72.png}
		}
	\subfigure{%
		\includegraphics[scale=0.36]{/mnt/Disk2/Brian_Curtis_042413_2/Results/images/Bz_File72.png}
		}
	\subfigure{%
		\includegraphics[scale=0.36]{/mnt/Disk2/Brian_Curtis_042413_3/Results/images/Bz_File72.png}
		}
	\caption{$B_z$ for OpenGGCM (top), BATS-R-US (middle), and SWMF (bottom).}
	\figSpace
	\label{fig:BHighCompressionEnd}
\end{figure}

\begin{figure}
	\centering
	\subfigure{
		\includegraphics[scale=0.36]{/mnt/Disk2/Results/8_9/images/Bz_Diff_File72.png}
		}
	\subfigure{
		\includegraphics[scale=0.36]{/mnt/Disk2/Results/8_10/images/Bz_Diff_File72.png}
		}
    \subfigure{
		\includegraphics[scale=0.36]{/mnt/Disk2/Results/9_10/images/Bz_Diff_File72.png}
		}
    \caption{$B_z$ percent differences between OpenGGCM and BATS-R-US (top),
    OpenGGCM and SWMF (middle), and BATS-R-US and SWMF (bottom).
    }
    \label{fig:BDiffHighEnd}
	\figSpace
\end{figure}


% RHO%
With a dense and fast solar wind, a large region of high density at the
magnetopause is expected, consistent with Figure
\ref{fig:rhoHighCompressionBeginning}. Also, as described in the first
experiment, the OpenGGCM model does not include a model of the inner magnetosphere while
the SWMF model does, and the observations are consistent with this.
In Figure \ref{fig:rhoHighCompressionBeginning}, the top two plots show the BATS-R-US and
SWMF models to have higher densities in the current sheet region compared to the
OpenGGCM, while the bottom plot shows that the densities in the current sheet are similar
between the BATS-R-US and SWMF models. 
The scalar $\rho$ plots do not show many differences from the beginning to the end of the run.

% Reads as if a plot be effected by a response.  Not sure what you are trying to say.   

%The difference plots towards the end of
%the run are effected by the three different responses by the three models for
%high compression. 
\begin{figure}
	\centering
	\subfigure{%
		\includegraphics[scale=0.36]{/mnt/Disk2/Brian_Curtis_042413_1/Results/images/rho_File1.png}
		}
	\subfigure{%
		\includegraphics[scale=0.36]{/mnt/Disk2/Brian_Curtis_042413_2/Results/images/rho_File1.png}
		}
	\subfigure{%
		\includegraphics[scale=0.36]{/mnt/Disk2/Brian_Curtis_042413_3/Results/images/rho_File1.png}
		}
	\caption{$\rho$ for OpenGGCM (top), BATS-R-US (middle), and SWMF (bottom).}
	\figSpace
	\label{fig:rhoHighCompressionBeginning}
\end{figure}

\begin{figure}
	\centering
	\subfigure{
		\includegraphics[scale=0.36]{/mnt/Disk2/Results/8_9/images/rho_Diff_File1.png}
		}
	\subfigure{
		\includegraphics[scale=0.36]{/mnt/Disk2/Results/8_10/images/rho_Diff_File1.png}
		}
    \subfigure{
		\includegraphics[scale=0.36]{/mnt/Disk2/Results/9_10/images/rho_Diff_File1.png}
		}
    \caption{$\rho$ percent differences between OpenGGCM and BATS-R-US (top),
    OpenGGCM and SWMF (middle), and BATS-R-US and SWMF (bottom).
    }
    \label{fig:rhoDiffHighBeginning}
	\figSpace
\end{figure}


%U_X%
As shown in Figure \ref{fig:UxHighCompressionBeginning},
all three models begin with high tailward $U_x$ in the current sheet
region. The maximum $U_x$ observed in each model is different. The OpenGGCM
model (top) maximum $U_x$ is 1,410~$km/s$, the BATS-R-US model (middle) maximum $U_x$ is
1,560~$km/s$, while the SWMF model (bottom) maximum $U_x$ is 841~$km/s$. The SWMF
model maximum $U_x$ is just over half of the other two models.
\begin{figure}
	\centering
	\subfigure{%
		\includegraphics[scale=0.36]{/mnt/Disk2/Brian_Curtis_042413_1/Results/images/Ux_File1.png}
		}
	\subfigure{%
		\includegraphics[scale=0.36]{/mnt/Disk2/Brian_Curtis_042413_2/Results/images/Ux_File1.png}
		}
	\subfigure{%
		\includegraphics[scale=0.36]{/mnt/Disk2/Brian_Curtis_042413_3/Results/images/Ux_File1.png}
		}
	\caption{$U_x$ for OpenGGCM (top), BATS-R-US (middle), and SWMF (bottom).}
	\figSpace
	\label{fig:UxHighCompressionBeginning}
\end{figure}
At the end of the runs, as shown in Figure \ref{fig:UxHighCompressionEnd}, the
same movements occur in the BATS-R-US model (middle) and SWMF (bottom), and the same
region that was observed in the OpenGGCM model (top) is observed in $U_x$ plots as a
large region of Earthward velocity.
\begin{figure}
	\centering
	\subfigure{%
		\includegraphics[scale=0.36]{/mnt/Disk2/Brian_Curtis_042413_1/Results/images/Ux_File72.png}
		}
	\subfigure{%
		\includegraphics[scale=0.36]{/mnt/Disk2/Brian_Curtis_042413_2/Results/images/Ux_File72.png}
		}
	\subfigure{%
		\includegraphics[scale=0.36]{/mnt/Disk2/Brian_Curtis_042413_3/Results/images/Ux_File72.png}
		}
	\caption{$U_x$ for OpenGGCM (top), BATS-R-US (middle), and SWMF (bottom).}
	\figSpace
	\label{fig:UxHighCompressionEnd}
\end{figure}

As shown in Figure \ref{fig:UxDiffHighBeginning}, the largest differences are in
the current sheet region for all three comparisons. Between the OpenGGCM and
BATS-R-US models (top) the BATS-R-US model has largest differences in $U_x$,
with the OpenGGCM model having larger differences in the tail lobes. Between the OpenGGCM and
SWMF models (middle), the SWMF model has larger differences. Between the
BATS-R-US and SWMF models (bottom) the BATS-R-US model has the largest differences in $U_x$ in the
current sheet region with the SWMF model having larger differences in the tail
lobes of the distant tail.
\begin{figure}
	\centering
	\subfigure{
		\includegraphics[scale=0.36]{/mnt/Disk2/Results/8_9/images/Ux_Diff_File1.png}
		}
	\subfigure{
		\includegraphics[scale=0.36]{/mnt/Disk2/Results/8_10/images/Ux_Diff_File1.png}
		}
    \subfigure{
		\includegraphics[scale=0.36]{/mnt/Disk2/Results/9_10/images/Ux_Diff_File1.png}
		}
    \caption{$U_x$ percent differences between OpenGGCM and BATS-R-US (top),
    OpenGGCM and SWMF (middle), and BATS-R-US and SWMF (bottom).
    }
    \label{fig:UxDiffHighBeginning}
	\figSpace
\end{figure}
As time progresses, all three comparisons have differences similar to that observed
in the beginning of the run.  The differences between the BATS-R-US and SWMF
models have increased in the current sheet region giving the BATS-R-US model higher
values, as shown in Figure \ref{fig:UxDiffHighEnd}.
\begin{figure}
	\centering
	\subfigure{
		\includegraphics[scale=0.36]{/mnt/Disk2/Results/8_9/images/Ux_Diff_File56.png}
		}
	\subfigure{
		\includegraphics[scale=0.36]{/mnt/Disk2/Results/8_10/images/Ux_Diff_File56.png}
		}
    \subfigure{
		\includegraphics[scale=0.36]{/mnt/Disk2/Results/9_10/images/Ux_Diff_File56.png}
		}
    \caption{$U_x$ percent differences between OpenGGCM and BATS-R-US (top),
    OpenGGCM and SWMF (middle), and BATS-R-US and SWMF (bottom).
    }
    \label{fig:UxDiffHighEnd}
	\figSpace
\end{figure}

\subsubsection{Low Magnetospheric Compression}
This section contains a discussion of the differences between the three models for
conditions that lead to a low magnetospheric compression.
%B%
As shown in Figure \ref{fig:BLowCompressionBeginning}, $B_z$ in
the OpenGGCM model (top) shows the most stretching in the magnetotail, while the
BATS-R-US (middle) and SWMF (bottom) appear similar to each other and do not stretch as far.
\begin{figure}
	\centering
	\subfigure{%
		\includegraphics[scale=0.36]{/mnt/Disk2/Brian_Curtis_042413_5/Results/images/Bz_File1.png}
		}
	\subfigure{%
		\includegraphics[scale=0.36]{/mnt/Disk2/Brian_Curtis_042413_6/Results/images/Bz_File1.png}
		}
	\subfigure{%
		\includegraphics[scale=0.36]{/mnt/Disk2/Brian_Curtis_042413_7/Results/images/Bz_File1.png}
		}
	\caption{$B_z$ for OpenGGCM (top), BATS-R-US (middle), and SWMF (bottom).}
	\figSpace
	\label{fig:BLowCompressionBeginning}
\end{figure}
As time progresses, the OpenGGCM model shows the most fluctuations,
while the BATS-R-US and OpenGGCM models have minimal fluctuations. For all three
models, there is minimal change from the beginning of the run to the end of the
run.

As shown in Figure \ref{fig:BDiffLowBeginning}, the OpenGGCM and BATS-R-US
models (top) and the OpenGGCM and SWMF models (middle) differences are similar. The differences between the BATS-R-US and SWMF models are near
zero for all regions in the magnetosphere. There is only a small region of differences in the magnetopause
tailward of the cusps. As time progresses in the run, there are minimal changes
between all three models, although the OpenGGCM and BATS-R-US models have a difference
reduction in the current sheet region. The same is true for the OpenGGCM and
SWMF model differences.
\begin{figure}
	\centering
	\subfigure{
		\includegraphics[scale=0.36]{/mnt/Disk2/Results/12_13/images/Bz_Diff_File1.png}
		}
	\subfigure{
		\includegraphics[scale=0.36]{/mnt/Disk2/Results/12_14/images/Bz_Diff_File1.png}
		}
    \subfigure{
		\includegraphics[scale=0.36]{/mnt/Disk2/Results/13_14/images/Bz_Diff_File1.png}
		}
    \caption{$B_z$ percent differences between OpenGGCM and BATS-R-US (top),
    OpenGGCM and SWMF (middle), and BATS-R-US and SWMF (bottom).
    }
    \label{fig:BDiffLowBeginning}
	\figSpace
\end{figure}

%RHO%
The maximum density for the OpenGGCM, BATS-R-US and SWMF model runs are
11~$cm^{-3}$, 29~$cm^{-3}$ and 29~$cm^{-3}$ respectively. As shown in Figure
\ref{fig:rhoLowCompressionBeginning}, the near-Earth $\rho$ is high for the
BATS-R-US and SWMF models with the OpenGGCM model having low $\rho$. There is
minimal change between the three models for the length of the run.
\begin{figure}
	\centering
	\subfigure{%
		\includegraphics[scale=0.36]{/mnt/Disk2/Brian_Curtis_042413_5/Results/images/rho_File1.png}
		}
	\subfigure{%
		\includegraphics[scale=0.36]{/mnt/Disk2/Brian_Curtis_042413_6/Results/images/rho_File1.png}
		}
	\subfigure{%
		\includegraphics[scale=0.36]{/mnt/Disk2/Brian_Curtis_042413_7/Results/images/rho_File1.png}
		}
	\caption{$\rho$ for OpenGGCM (top), BATS-R-US (middle), and SWMF (bottom).}
	\figSpace
	\label{fig:rhoLowCompressionBeginning}
\end{figure}

As shown in Figure \ref{fig:rhoDiffLowBeginning}, the OpenGGCM vs. BATS-R-US
models (top) and the OpenGGCM vs. SWMF models (middle) show large differences
near Earth extending tailward in the current sheet region. The only differences between the BATS-R-US
and SWMF models (bottom) are near Earth where the BATS-R-US model has the larger
differences with the SWMF model having higher values in the distant tail lobes.
As time progresses, there are no major changes in the plots.
\begin{figure}
	\centering
	\subfigure{
		\includegraphics[scale=0.36]{/mnt/Disk2/Results/12_13/images/rho_Diff_File1.png}
		}
	\subfigure{
		\includegraphics[scale=0.36]{/mnt/Disk2/Results/12_14/images/rho_Diff_File1.png}
		}
    \subfigure{
		\includegraphics[scale=0.36]{/mnt/Disk2/Results/13_14/images/rho_Diff_File1.png}
		}
    \caption{$\rho$ percent differences between OpenGGCM and BATS-R-US (top),
    OpenGGCM and SWMF (middle), and BATS-R-US and SWMF (bottom).
    }
    \label{fig:rhoDiffLowBeginning}
	\figSpace
\end{figure}

% Ux%
The maximum $U_x$ for the OpenGGCM, BATS-R-US, and SWMF models are 809~$km/s$,
347~$km/s$ and 349~$km/s$, respectively. Shown in Figure
\ref{fig:UxLowCompressionBeginning}, the current sheet $U_x$ is high for the
OpenGGCM but not for the BATS-R-US and SWMF models. There is minimal change
between the three models for the length of the time series. The OpenGGCM model
has fluctuations in the tail region for $U_x$, while the other two models do not.
\begin{figure}
	\centering
	\subfigure{%
		\includegraphics[scale=0.36]{/mnt/Disk2/Brian_Curtis_042413_5/Results/images/Ux_File1.png}
		}
	\subfigure{%
		\includegraphics[scale=0.36]{/mnt/Disk2/Brian_Curtis_042413_6/Results/images/Ux_File1.png}
		}
	\subfigure{%
		\includegraphics[scale=0.36]{/mnt/Disk2/Brian_Curtis_042413_7/Results/images/Ux_File1.png}
		}
	\caption{$U_x$ for OpenGGCM (top), BATS-R-US (middle), and SWMF (bottom).}
	\figSpace
	\label{fig:UxLowCompressionBeginning}
\end{figure}

As shown in Figure
\ref{fig:UxDiffLowBeginning}, the OpenGGCM vs. BATS-R-US models (top) have higher
$U_x$ in the current sheet and near-Earth regions for the OpenGGCM model. The
OpenGGCM vs. SWMF models (middle) differences in $U_x$ are higher in the current
sheet and near-Earth regions for the OpenGGCM model. The BATS-R-US vs. SWMF
model differences show high values for the BATS-R-US model in the near-Earth tail
lobes of the magnetosphere.
\begin{figure}
	\centering
	\subfigure{
		\includegraphics[scale=0.36]{/mnt/Disk2/Results/12_13/images/Ux_Diff_File1.png}
		}
	\subfigure{
		\includegraphics[scale=0.36]{/mnt/Disk2/Results/12_14/images/Ux_Diff_File1.png}
		}
    \subfigure{
		\includegraphics[scale=0.36]{/mnt/Disk2/Results/13_14/images/Ux_Diff_File1.png}
		}
    \caption{$U_x$ percent differences between OpenGGCM and BATS-R-US (top),
    OpenGGCM and SWMF (middle), and BATS-R-US and SWMF (bottom).
    }
    \label{fig:UxDiffLowBeginning}
	\figSpace
\end{figure}

\subsection{Discussion and Conclusions}
\subsubsection{High Compression}

A strong southern component of the solar wind IMF will weaken the magnetic
field at Earth's magnetopause due to their different orientations. Combined with a
fast solar wind velocity and high solar wind densities, the magnetosphere will
compress because the kinetic pressure from the solar wind
becomes larger than the magnetic pressure of Earth's magnetic field.

The $B_z$ plots show magnetopause locations. The OpenGGCM model
is closest to Earth due to the weaker magnetic pressure that is a result of the model
not accounting for a ring current. The BATS-R-US magnetopause, because it does
not account for the ring current, is expected to be similar to the OpenGGCM. This is not
observed in the results, and is likely due to differences in how each model
couples to the inner boundary. The SWMF magnetopause location
is expected to be farther from Earth than that of the OpenGGCM model.

The three models produce three different predictions of how the
magnetosphere reacts to solar wind conditions that will cause high compression. The OpenGGCM model has a region of higher $B_z$ in the near-Earth
current sheet region at the end of the run. There are high Earthward velocities
in the same region at the end of the run.  The velocities in the tail
region come from reconnection in the current sheet. The BATS-R-US model appears to stop movement entirely before the midway point
of the run. The reasoning for this result is currently unknown and requires
further research.  The SWMF model, as time progresses, appears to oscillate with increasing
movement not stopping by the end of the run. One hypothesis is that there is a
destabilization caused by the RCM in which the global magnetosphere starts to
oscillate. Towards the end of the run there is little increase in the
oscillations, and along with the OpenGGCM model results, it would be important to
study the model results with a longer time frame than 6 hours.

\subsubsection{Low Compression}
Opposite to the conditions for a high compression of the magnetosphere, low
compression occurs with a slow solar wind $U_x$, a
positive $B_z$ and a low $\rho$. $B_z^{IMF}$ and Earth's $B_z$ are both the
same direction, which increases the magnetic field strength at the magnetopause
causing a strong magnetic pressure that pushes the magnetopause Sunward. The
effect of the ring current is minimal, which keeps the magnetopause locations
fairly close to one another for all three models.

The OpenGGCM model current sheet velocities are larger in a low compression
environment than that in the BATS-R-US and SWMF models, where the differences were very
close to zero over the entire domain. The speeds of plasma in the current
sheet can play a large role in the effects of the plasma as observed at Earth.
The resistive MHD used in the OpenGGCM model may allow for faster reconnection in the current sheet region and explain the faster
velocities observed in the current sheet region.

\subsubsection{Summary}
For extreme conditions in the solar wind, the
following occurs in the considered MHD models of the magnetosphere:
\begin{itemize}
  \item The OpenGGCM model has a large region of Earthward $U_x$ in the current
  sheet region that grows as time progresses in a compressed environment.
  \item The BATS-R-US model is either completely stable or stops in a compressed
  environment.
  \item In a compressed environment, the SWMF model will eventually oscillate.
  \item The OpenGGCM model has the highest tailward velocities under strong compression conditions.
  \item The RCM inner magnetosphere model may explain the smaller maximum velocities observed in the SWMF model.
  \item The OpenGGCM model has the highest $B_z$ under strong compression.
  \item All three models have similar magnetopause positions under low compression.
  \item The OpenGGCM model current sheet velocities are largest under low compression.
\end{itemize}

\chapter{Summary and Conclusions}

The objective of this work was to perform three experiments in order to further
our understanding of the behavior of and differences between magnetospheric MHD
models. Three experiments were performed.  The first experiment was used to
determine the differences between model predictions when $B_z^{IMF}$ changed
from positive to negative while all other inputs were constant.  The second
experiment determined the sensitivity of the models to the length of time that
they were preconditioned prior to a change in $B_z^{IMF}$ from positive to
negative.  The third experiment used extreme solar wind conditions,
corresponding to weak and strong magnetospheric compression.

The type of analysis performed for this thesis is expected to be
useful to model developers. A next step for this research will be to
expand the experiments to include more magnetospheric models. We found
that the model output depended on preconditioning time; therefore a
next step is to determine the shortest preconditioning time for which
the output is nearly the same.  Another analysis will be to
expand the number of artificial input conditions used for the comparisons made in the first experiment. 

Finally, an important goal for model developers is to allow forecasters to have
and understanding of the uncertainties and differences between the predictions of their models. To
accomplish this, the tendencies determined from this thesis should be
used when interpreting forecasts.
We conclude that more validation of the type performed for this thesis is
needed because of the significant differences found between models (and within a given model for different preconditioning times) and the fact that output of these models are regularly used for interpretation of observations.

